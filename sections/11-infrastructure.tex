\section{Infrastructure}
\label{sec:infrastructure}

\subsection{Release Cycle, Versioning, Long-term Support}
As of the release of 1.0, a formal release schedule for \sunpypkg has been adopted.
Two releases are planned per year with 6 months between each.
In order to align with the release cycle of \astropypkg, a major upstream package, the plan is to release each May and November.
The first release of the year will be a Long Term Support (LTS) release which will be supported for 12 months or until the next LTS release.
The second release will be a non-LTS and will be supported for 6 months or until the next release.
A formal versioning convention has also been adopted going forward.

\sunpypkg will follow the following versioning system: ``X.Y.z", where the three components have the following meaning;
``X" is the LTS version number, which will be incremented with every LTS,
``Y" is the release counter, this will be 0 for LTS releases and increment for each intermediate non-LTS release,
``z" is the bug fix number, and is to be incremented for any bug-fix releases.

The primary goal of the adoption of this formal release structure is to provide clarity for support of releases as well as improving predictability of codebase changes for users of \sunpypkg.

\subsection{Continuous Integration}
The \sunpyproj follows common practices and makes extensive use of continuous integration which provide automated testing and code change inspection.
All proposed code contributions trigger test suites to be run on a number of free services (e.g.
\href{https://azure.microsoft.com/en-gb/services/devops/pipelines/}{Microsoft's Azure pipelines},
\href{https://circleci.com}{circleci},
\href{https://codecov.io/}{Codecov}) which integrate into the \github website.
These services provide the first review of any contribution by running the test suite on each operating system (Windows, Mac, Linux), testing the documentation build, running figure tests, and providing code coverage metrics.
Additionally \href{https://travis-ci.org}{Travis CI} is used to run the entire test suite on a daily cadence.
\sunpypkg's test suite can be broken down into three broad categories: offline, online and figure tests.
Offline test suite checks the majority of the codebase.
Online tests specifically tests code that make use of online data providers (e.g. VSO or JSOC).
These tests depend on the availability of these online services.
Finally, figure tests generate plots and issue failures if they have changed.
This enables high level functionality testing.
These tests coupled with these services are an important tool for maintaining the integrity of the package and makes it easier for new contributors to understand the impact of their changes.

\subsection{Documentation and Gallery}
The \sunpyproj strives to provide up to date, approachable and high quality documentation.
All documentation for \sunpypkg as well as all affiliated packages is written using the commonly-used \code{Sphinx} documentation build system.
This system supports using plain text files with a markup language called \code{reStructuredText} (RST).
The build process converts these files including documentation strings in Python files into HTML, PDF, or LATEX documents.
We make use of the \code{sphinx-gallery} extension to build a gallery of analysis examples as well as the extension \code{sphinx-automodapi} which generates documentation pages that list all of the available classes, functions, and attributes.
Our \href{http://docs.sunpy.org/en/stable/}{online documentation} is automatically built and hosted on \href{https://readthedocs.org/}{Read the Docs} for all releases.

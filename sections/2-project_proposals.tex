\section{Project Organization \& Enhancement Proposals}
\label{sec:proj_org}

The organization of the \sunpyproj is modeled on the structure of a board-only not-for-profit corporate entity.
It consists of an up-to 10 member self-selected board.
An executive director, elected by the board, leads the core development team and the development of \sunpypkg core package, provides user support, and supports the development of affiliated packages.
As such, the executive director is also referred to as the lead developer.
A deputy lead developer and release manager, as well as other volunteers from the developer community, support the lead developer.
Board members serve two year terms while the lead developer serves one year terms with no term limits.

The \sunpyproj is formally defined through SunPy Enhancement Proposals (SEPs) which are modeled after the Python Enhancement Proposal process\footnote{\url{https://www.python.org/dev/peps/}}.
All SEPs are version-controlled as well as publicly available\footnote{\url{https://github.com/sunpy/sunpy-SEP}}.
The first two SEPs (SEP-0001 and SEP-0002) define what is an SEP as well as the SunPy organization.

SEPs are used to both define the organization of the project as a whole as well as technical requirements for the \sunpypkg core package and affiliated packages.
There are generally three types of SEPs.
\begin{itemize}
    \item \textbf{Standard}: Introduces and describes a new feature, or changes to an existing feature, and is meant to function as a high-level technical design document.
    \item \textbf{Process}: Describes a new process, or changes to an existing process, in the organization.
    \item \textbf{Informational}: Provides information and does not introduce any new features, changes, or processes.
\end{itemize}

As of the time of writing, there are a total of 8 SEPs.
Some notable SEPs led to the adoption of physical units throughout the code base (SEP-0003, see Section~\ref{sec:units}), defined the affiliated package program (SEP-0004, see Section~\ref{sec:affil_package}), standardized the use of coordinate and coordinate transformations (SEP-0005, see Section~\ref{sec:coords}), and led to the adoption of a high precision scientific time format (SEP-0008).

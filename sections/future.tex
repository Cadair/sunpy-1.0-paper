\section{Conclusion - Steven}
\label{sec:conclusion}
% edited by Steven C. 25-Apr-2019
% sdc still needs significant work...

Development of the \sunpypkg core package has been going on now for 8 years with the project adopting a formal organization to lead the project 5 years ago. 
In that time, the core package has grown to provide significant functionality for a growing number of users. 
Significant additional features are currently missing and either under active development on are planned for development. 
These include support for spectra and spectral fitting, support for multi-dimensional datasets (e.g. slit spectrographs, radio spectrograms), providing a standardized approach to metadata.
The board structure is meant to ensure that the leadership of the project remains within the community and reflects a broad spectrum of the community.
The project has adopted an official Code of Conduct\footnote{\url{http://docs.sunpy.org/en/stable/coc.html}} whose purpose is to ensure that the \sunpy community is positive, inclusive, successful, and growing. 
The three primary aspects that are encouraged in each community member is to be open, considerate and respectful. 
A significant obstacle to the growth of the developer community has been the difficulty in providing the appropriate tools and guidance to users to contribute to \sunpypkg. 
The conversion from user to contributor requires significant additional skills which are not prevalent in the solar community including knowledge of version control, writing user documentation and unit testing. 
This is in addition to the not insignificant effort of refactoring code for public use. 
Finally, informal surveying suggests that the scope of the \sunpypkg package is not clear which likely disincentive contributions. 
The \sunpyproj is now a member of the Python in Heliophysics Community (PyHC)\footnote{\url{heliopython.org}}, whose members contribute to a collection of over fifty Python packages that span every sub-discipline within heliophysics which includes solar physics. 

\section{acknowledgements}
We would like to acknowledge financial contributions by Google as part of the Google Summer of Code program.
\section{Physical Units - Bobra}
\label{sec:units}
% edited by Steven C. 23-Apr-2019

%Discuss the units SEP. discuss that we are using astropy units and provide a short description of it's capabilities. mention that sunpy.sun provides quantities with units. mention the NASA units disaster and reference units policy for NASA as reference here is a interesting reference \url{https://sma.nasa.gov/news/safety-messages/safety-message-item/lost-in-translation}

Calculations of physical quantities have traditionally been performed in software using raw numbers in the interest of speed as well as simplicity. 
Physical units have frequently been recorded in comments or in documentation which can easily lead to errors or even disaster. 
The Mars Climate Orbiter mishap in 1988 was caused by a unit discrepancy. 
The spacecraft trajectory was reported in English units instead of metric which led to the the Mars Climate Orbiter entering the Martian atmosphere well-below its intended altitude causing complete mission failure \citep{mco_mishap_report}. 
A more modern approach which ensures that physical quantities are fully described in 
software is described by \citep{Damevski2009}. 
The \sunpypkg implements this concept in all of its code through functionality provided by \astropy. 
The \code{astropy.units} subpackage implements support for physical quantities by extending \numpy array objects and define a \code{Quantity} object which consists of a number with associated units. 
Tests have shown that the overhead to using this functionality are minimal in most cases. 
It is formally mandated that all user-facing functionality provided by \sunpypkg make use of \code{Quantity} objects if appropriate. 
\code{astropy.units} also provides a tool to restrict the input in function definition to units of a particular type (e.g. length, mass). 
This is used through all \sunpypkg modules.

The \package{sunpy.sun} module contains constants, parameters and models of the Sun provided which are all provided as physical quantities as \code{astropy.units.Constants} which are \code{Quantity} with additional reference metadata. 
These include variable quantities, such as the Carrington rotation number, as well as constants, such as the solar mass. 

\todo{do we want to discuss astropy time in this section as well?}


\section{Support \& Sustainability}
\label{sec:support}

To date, the \sunpyproj relies largely on unpaid, volunteer efforts from early career scientists.
The project has not received any significant direct financial support for its work facilitating and promoting open-source and open development software including \sunpypkg itself.
This presents a number of challenges which the \astropy community also faces and are well described in \cite{PriceWhelan:2018ji} and \cite{Muna2016}.

The National Academies of Sciences, Engineering, and Medicine's report on Open Source Software Policy Options for NASA Earth and Space Sciences \citep{NAP2018} outlines several solutions to alleviate these problem -- namely that the NASA Science Mission Directorate provide funding for new and existing open source software projects, promote scientists who spend time developing and improving open source software projects, and offer prizes for exemplary contributions to the open source software community.
The National Academies of Sciences, Engineering, and Medicine's report on Reproducibility and Replicability in Science \citep{NAP2019} recommends that funding agencies invest in the research and development of open-source software that support reproducibility.
The SunPy community supports solutions like these for all relevant funding agencies and furthermore has the ability to accept financial contributions from institutions or individuals through the NumFOCUS organization, a 501(c)(3) public charity which collects and manages tax-deductible contributions for many open-source scientific software packages such as \numpy and \astropy.

The \sunpyproj does participate in two summer of code programs, which offer students stipends to write code for open-source projects, via the OpenAstronomy umbrella organization.
Since 2013, 16 students contributed to \sunpypkg and affiliated packages through the Google Summer of Code (GSoC)\footnote{\url{https://summerofcode.withgoogle.com/}} program.
Since 2011, 7 students contributed through a similar program funded by the European Space Agency called The ESA Summer of Code in Space\footnote{\url{https://socis.esa.int/}}.
Many SunPy community members served as mentors for these students\footnote{\url{https://github.com/sunpy/sunpy/wiki/Wall-of-Fame}}.
In 2018, s GSoC student Vishnunarayan K. I. was presented an award for exceptional contributions to the open source scientific software ecosystem for his efforts to update \sunpypkg with a modern astronomical time system by NumFOCUS.
